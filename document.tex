\documentclass{bioinfo}
\copyrightyear{2005}
\pubyear{2005}

\begin{document}
\firstpage{1}

\title[short Title]{Brain: Biomedical Knowledge Manipulation}
\author[Sample \textit{et~al}]{Samuel Croset\,$^{1,\footnote{to whom correspondence should be addressed}}$, JPO\,$^{2}$ and DRS\,$^{2}$}
\address{$^{1}$Department of XXXXXXX, Address XXXX etc.\\
$^{2}$Department of XXXXXXXX, Address XXXX etc.}

\history{Received on XXXXX; revised on XXXXX; accepted on XXXXX}

\editor{Associate Editor: XXXXXXX}

\maketitle

\begin{abstract}

\section{Summary:}
This section should summarize the purpose/novel features of the program in one or two sentences.

\section{Availability and Implementation:}
This section should state software availability if the paper focuses mainly on 
software development or on the implementation of an algorithm. Examples are:
Freely available on the web at http://www.example.org. Website implemented in Perl, MySQL and 
Apache, with all major browsers supported'; or 'Source code and binaries freely available for download at URL, 
implemented in C++ and supported on linux and MS Windows'. The complete address (URL) should be given. If the manuscript 
describes new software tools or the implementation of novel algorithms the software must be freely available to non-commercial 
users. Authors must also ensure that the software is available for a full TWO YEARS following publication. The editors of 
Bioinformatics encourage authors to make their source code available and, if possible, to provide access through an open source 
license see www.opensource.org for examples.


\section{Contact:}
\href{croset@ebi.ac.uk}{croset@ebi.ac.uk}

\section{Supplementary information:}
Links to additional figures/data available on a web site, or 
reference to online-only Supplementary data available at the journal's web site.

\end{abstract}
Relational databases hold most of the available structured biomedical information. The content of these repositories is often
extracted from scientific literature by manual curation with the help of text-mining tools. The transformation from raw text into
structured data is most important, as the curated information can then be classified, managed and queried more easily. Databases facilitate
the re-use of previous work in a computer-friendly manner and support the biomedical knowledge to scale-up. In order to leverage further 
more the existing information, the current trend is at data integration and interoperability, with large projects such as ELIXIR leading the way.
The underlying idea assumes that increasingly complex biomedical challenges such as finding new treatments for diseases could be addressed 
by combining the content of independant repositories via the Internet. 
Traditional relational databases are however an obstacle to realize this vision, mostly because of their lack of support for interoperability:
The schema structuring the data is indeed very repository-specific which limits the combination of the native content with external data in an
efficient and meaningful fashion. In order to address this issue, a series a standard forming the semantic web have been developped. One of them,
the Resources Description Framework (RDF) enables the exposition of the underlying structure as part of the data themselves. This
representation relies on triples as building block, composed as \emph{subject - relation - object} and where 
the types of the data are identified with a special relation called \emph{rdf:type}. More complicated data structures, such as sub-classes or
transitive relations can be further expressed via another standard, the Web Ontology Language (OWL). OWL derives from Description logic and
is used to capture the knowledge of a domain of interest in the form of a structured vocabulary. This feature makes it particularly 
interesting from the point of view of life science, as a lot of ontologies and classification have been devellopped since the origin of 
the discipline. OWL is often expressed in combination with RDF data in order to reveal the underlying schema, but 
it can also be used as such as an implemetation of a part of description logic. 
Knowledge bases can therefore be built without being necessarly expressed as RDF triples while still preserving all the advantages in regards 
to data integration and interoperability.
Brain, the library presented in this manuscript aims at facilitating the contruction and manipulation of 
such knowledge bases, rather than being oriented towards the consumption of RDF data. We will present first the biomedical motivation for 
the particular subset of OWL supported by Brain so called OWL 2EL. The will be discussed the main features of Brain in regards to this profile.

what is owl 2el?
owl 2el is a subset of owl that is focused on trackacle profile of owl.
what's a trackable profile?
axioms that are not too complicated to understand and that are fast to compute.
What are axioms?
axioms are one of the fundamental block of description logic, it's assertion about facts present in the knowledge base. A reasoner
is capable of understanding them and conclude new facts from it, it implicit knowledge retrieval.
what's a reasoner?
reasoner is a tool that is going to classifiy an ontology based on the type of constructs faced. it is also used to query the ontology or to
check the consistency.
Why is OWL 2 EL interesting for us?
most biomedical ontologies are following this profile (citations).
Biomedical knowledge representation resolves around classes more than instances, as opposite to the RDF or relational representation,
which is in the scope of OWL 2el.
KB are decomposed in tbox and abox and rbox, classes belong to the tbox, which is large in biology.

how does one implement a knowledge base?
need to use the OWL-API and a specific reasoner or Protege
ok, what's the problem with that?
extremlely conprehensive solution for the API but heavy. Protege is a good GUI but not suitable for programatic access.
Need for a lightweight solution specific for OWL 2 EL, in order to build web service and OWL end-points.
ok, what is the solution?
Brain, a facade for the owl api and the ELk reasoner.
what are the main features of brain?
brain deals with one ontology at a time only
focusing on the biology, less the computer science.
brain deals only with unique names non redundant names for entites
string based interaction to build classes and class expression using the intuitive manchester syntax
error-handling driven
convenience method for import/export of stuff, management of prefixes
list of the contruct and implementation as table (copy pasta from website).
scalable because of Elk.
creration and iomport of external ontologies (learn)
consistency checking

Why is brain interesting for the community
tool to build KB scalable solution in order to preapre the next stage of data integration with KB reasoner driven and semantic web.

\section*{Acknowledgement}
These should be included at the end of the text and not in footnotes. Please ensure you acknowledge all 
sources of funding, see funding section below.
Details of all funding sources for the work in question should be given 
in a separate section entitled 'Funding'. This should appear before the 'Acknowledgements' section.

\paragraph{Funding\textcolon}
The following rules should be followed:
The sentence should begin: This work was supported by …
 
The full official funding agency name should be given, i.e. National Institutes of Health, not NIH 
(full RIN-approved list of UK funding agencies) Grant numbers should be given in brackets as follows: grant number xxxx 
Multiple grant numbers should be separated by a comma as follows: grant numbers xxxx, yyyy 
Agencies should be separated by a semi-colon (plus ‘and’ before the last funding agency) 
Where individuals need to be specified for certain sources of funding the following text should be added after the
relevant agency or grant number to author initials.
Oxford Journals will deposit all NIH-funded articles in PubMed Central. See Depositing articles in repositories – 
information for authors for details. Authors must ensure that manuscripts are clearly indicated as NIH-funded using the guidelines above.

\bibliographystyle{natbib}
\bibliographystyle{achemnat}
\bibliographystyle{plainnat}
\bibliographystyle{abbrv}
\bibliographystyle{bioinformatics}

\bibliographystyle{plain}

\bibliography{document}

\end{document}
